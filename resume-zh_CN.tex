% !TEX TS-program = xelatex
% !TEX encoding = UTF-8 Unicode
% !Mode:: "TeX:UTF-8"

\documentclass{resume}
% \usepackage{zh_CN-Adobefonts_external} % Simplified Chinese Support using external fonts (./fonts/zh_CN-Adobe/)
\usepackage{NotoSansSC_external}
% \usepackage{NotoSerifCJKsc_external}
% \usepackage{zh_CN-Adobefonts_internal} % Simplified Chinese Support using system fonts
\usepackage{linespacing_fix} % disable extra space before next section
\usepackage{cite}

\begin{document}
\pagenumbering{gobble} % suppress displaying page number

\name{李冠宇}

\basicInfo{
  \email{liguanyubest@gmail.com} \textperiodcentered\ 
  \phone{(+86) 131-6257-8889} \textperiodcentered\ 
  \github[github/liguanyu]{https://github.com/liguanyu}}
 
\section{\faGraduationCap\  教育背景}
\datedsubsection{\textbf{上海交通大学}, 上海}{2017 -- 至今}
\textit{在读硕士研究生}\ 软件工程, 预计 2020 年 3 月毕业
\datedsubsection{\textbf{上海交通大学}, 上海}{2013 -- 2017}
\textit{学士}\ 机械工程及自动化

\vspace{1.5ex}
\section{\faUsers\  项目经历}
\datedsubsection{\textbf{Multi-OSv} }{2018年7月 -- 至今}
\role{C/C++,Unikernel}{独自负责}
Unikernel的多进程支持与隔离,基于\href{https://http://osv.io/}{\emph{OSv}}
% \begin{onehalfspacing}
\begin{itemize}
  \item 设计了Unikernel内部的三层权限隔离方案
  \begin{itemize}
    \item 灵活利用x86硬件特性(MPK),保证速度与隔离
    \item 利用Assembler等技术改造系统调用,保证控制流安全
    \item 系统调用速度是Linux syscall的1到2倍
  \end{itemize}
  \item 设计并实现了多进程支持特性,API实现(fork等)
  \begin{itemize}
    \item 重新设计地址空间及系统架构,改造线程调度
    \item fork的速度是Linux的3倍
  \end{itemize}
  \item 项目接近尾声,计划论文投稿后开源
\end{itemize}
% \end{onehalfspacing}

\datedsubsection{\textbf{TZ-NVM}}{2018年2月 -- 2018年6月}
\role{C,ARM,Linux}{负责整体设计与驱动开发}
% \begin{onehalfspacing}
利用ARM TrustZone加速Ext4文件系统日志
\begin{itemize}
  \item 利用TrustZone维护独立内存,抽象为非易失存储(NVM)
  \item 利用Linux驱动将TZ-NVM注册为块设备,供加速日志
  \item 在O\_SYNC模式下,写速度是普通Ext4分区的1.5倍
\end{itemize}
% \end{onehalfspacing}

\datedsubsection{\textbf{Deformable Leaf}}{2017年11月 -- 2018年1月}
\role{C++,OpenGL}{个人项目, 《高级计算机图形学》课程作业}
% \begin{onehalfspacing}
可变形的叶片建模,https://github.com/liguanyu/Deformable\_Leaf\_from\_A\_Picture
\begin{itemize}
  \item 利用OpenCV对叶片照片进行边缘分析,建立网格模型
  \item 利用Bezier曲线构建可灵活变形的叶脉
  \item 建立叶脉与叶面节点的关联模型,叶脉变形带动叶面变形
\end{itemize}
% \end{onehalfspacing}

% Reference Test
%\datedsubsection{\textbf{Paper Title\cite{zaharia2012resilient}}}{May. 2015}
%An xxx optimized for xxx\cite{verma2015large}
%\begin{itemize}
%  \item main contribution
%\end{itemize}

\vspace{1.5ex}
\section{\faCogs\  IT 技能}
% increase linespacing [parsep=0.5ex]
\begin{itemize}[parsep=0.5ex]
  \item 编程语言: 熟悉C,了解C++
  \item 平台: Linux
  \item 领域: Unikernel,Linux Kernel
\end{itemize}

\section{\faCheckSquare\  获奖情况}
\datedline{上海交通大学学业进步奖学金}{2014年 - 2015年}
\datedline{\textit{二等奖}, 第31届全国部分地区大学生物理竞赛(上海赛区)}{2014年}

% \section{\faInfo\ 其他}
% % increase linespacing [parsep=0.5ex]
% \begin{itemize}[parsep=0.5ex]
%   \item 2015年上海交通大学学业进步奖学金
%   \item 语言: 英语 - 熟练(TOEFL 89)
% \end{itemize}

%% Reference
%\newpage
%\bibliographystyle{IEEETran}
%\bibliography{mycite}
\end{document}
